% % % % % % % % % % % % % % % % % % % % % % % % % % % % % % % %
%                                                             %
% Treshur Allen                                               %
% ECE 351 - 52                                                %
% Lab 4                                                       %
% September 30, 2021                                          %
% This file contains the tex file for lab 4 with report and   %
% questions.                                                  %
% % % % % % % % % % % % % % % % % % % % % % % % % % % % % % % %

\documentclass{article}
\usepackage[utf8]{inputenc}


\title{ECE351 Lab 4 Report}
\author{Treshur Allen }
\date{September 30, 2021}

\begin{document}

\pagebreak

\maketitle

\section{Part 1:}

\subsection{Description:}

\begin{verbatim}
Implement the code we previously developed to create new transfer functions 

\end{verbatim}

\subsection{Task 1 Code:}

\begin{verbatim}
def stepFunc(t):
   
    u = np.zeros(t.shape)
    
    for i in range(len(t)):
        if t[i] <= 0:  
            u[i] = 0
        else:
            u[i] = 1
    return u

def h1(t):
    h = np.exp(-2*t) * (stepFunc(t) - stepFunc(t - 3))
    return h

def h2(t):
    h = stepFunc(t - 2) - stepFunc(t - 6)
    return h

def h3(t):
    h = np.sin((2*math.pi* 0.25) * t) * stepFunc(t)
    return h
\end{verbatim}

\subsection{Explanation:}

\begin{verbatim}
    This code from task one part 1 is the user defined functions I 
    created of the transfer functions defined in the lab manual.
\end{verbatim}

\maketitle

\section{Part 2:}

\subsection{Description:}

\begin{verbatim}
The purpose of this part of the lab was to show the
convolution of our new transfer functions and implementing the 
convolution function we created previously.

\end{verbatim}

\subsection{Code for plots and convolution function:}

\begin{verbatim}
   #---------------------PLOTS-------------------------

plt.figure(figsize = (10, 10))
plt.subplot(3, 1, 1)
plt.plot(t, h1(t))
plt.grid()
plt.ylabel('H1')
plt.title('Part 1 Task 2')


plt.subplot(3 , 1, 2)
plt.plot(t, h2(t))
plt.grid()
plt.ylabel('H2')

plt.subplot(3 , 1, 3)
plt.plot(t, h3(t))
plt.grid()
plt.ylabel('H3')
plt.xlabel('t')
#plt.show()

#-----------PART 2 FUNCTIONS/CONVOLUTION ----------------
steps = 1e-2 #step size
time = np.arange(2*t[0], 2*t[len(t)-1]+steps, steps)

def conv(f1,f2):
  Nf1 = len(f1)
  Nf2 = len(f2)
  f1Extended = np.append(f1,np.zeros((1,Nf2-1)))
  f2Extended = np.append(f2,np.zeros((1,Nf1-1))) 
  #these append function make it so the arrays have the same number
  #of elements to avoid errors 
  result = np.zeros(f1Extended.shape) #shape is the array

  for i in range(Nf2+Nf1-2):
        result[i] = 0
        for j in range(Nf1):
            if(i-j+1>0):
                try:
                    result[i] += f1Extended[j]*f2Extended[i-j+1]
                except:
                        print(i,j)
  return result


h1 = h1(t)
h2 = h2(t)
h3 = h3(t)
u = stepFunc(t)

#scipy.signal.convolve(h1,u)


plt.figure(figsize = (10, 10))
plt.subplot(3, 1, 1)
plt.plot(time, conv(h1, u))
plt.grid()
plt.ylabel('H1 Convolution')
plt.title('Part 2 Task 1')
#plt.show()

plt.subplot(3 , 1, 2)
plt.plot(time, conv(h2, u))
plt.grid()
plt.ylabel('H2 Convolution')
plt.xlabel('t')

plt.subplot(3 , 1, 3)
plt.plot(time,conv(h3, u))
plt.grid()
plt.ylabel('H3 Convolution')
plt.xlabel('t')
#plt.show()

\end{verbatim}

\subsection{Explanation:}

\begin{verbatim}
The code above has the plots I've created to show the convolutions. 
The main thing to focus on in this code is having to define a new time variable 
this was needed because the arrays of the convolution functions are doubled therefore 
in order to plot, our functions need to have the same sized array.
\end{verbatim}

\pagebreak

\maketitle

\section{Questions:}

\subsection{Leave any feed back:}

\begin{verbatim}

\end{verbatim}


\end{document}
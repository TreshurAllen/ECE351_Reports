% % % % % % % % % % % % % % % % % % % % % % % % % % % % % % % %
%                                                             %
% Treshur Allen                                               %
% ECE 351 - 52                                                %
% Lab 7                                                       %
% October 21, 2021                                            %
% This file contains the tex file for lab 7 with report and   
% questions.                                                  %
% % % % % % % % % % % % % % % % % % % % % % % % % % % % % % % 

%%%%%%%%%%%%%%%%%%%%%%%%%%%%%%%%%%%%%%%%%%%
%%% DOCUMENT PREAMBLE %%%
\documentclass[12pt]{report}
\usepackage[english]{babel}
%\usepackage{natbib}
\usepackage{url}
\usepackage[utf8x]{inputenc}
\usepackage{amsmath}
\usepackage{graphicx}
\graphicspath{{images/}}
\usepackage{parskip}
\usepackage{fancyhdr}
\usepackage{vmargin}
\usepackage{listings}
\usepackage{hyperref}
\usepackage{xcolor}

\definecolor{codegreen}{rgb}{0,0.6,0}
\definecolor{codegray}{rgb}{0.5,0.5,0.5}
\definecolor{codeblue}{rgb}{0,0,0.95}
\definecolor{backcolour}{rgb}{0.95,0.95,0.92}

\lstdefinestyle{mystyle}{
    backgroundcolor=\color{backcolour},   
    commentstyle=\color{codegreen},
    keywordstyle=\color{codeblue},
    numberstyle=\tiny\color{codegray},
    stringstyle=\color{codegreen},
    basicstyle=\ttfamily\footnotesize,
    breakatwhitespace=false,         
    breaklines=true,                 
    captionpos=b,                    
    keepspaces=true,                 
    numbers=left,                    
    numbersep=5pt,                  
    showspaces=false,                
    showstringspaces=false,
    showtabs=false,                  
    tabsize=2
}
 
\lstset{style=mystyle}

\setmarginsrb{3 cm}{2.5 cm}{3 cm}{2.5 cm}{1 cm}{1.5 cm}{1 cm}{1.5 cm}

\title{1}								
% Title
\author{ Treshur Allen}						
% Author
\date{October 21, 2021}
% Date

\makeatletter
\let\thetitle\@title
\let\theauthor\@author
\let\thedate\@date
\makeatother

\pagestyle{fancy}
\fancyhf{}
\rhead{\theauthor}
\lhead{\thetitle}
\cfoot{\thepage}
%%%%%%%%%%%%%%%%%%%%%%%%%%%%%%%%%%%%%%%%%%%%
\begin{document}

%%%%%%%%%%%%%%%%%%%%%%%%%%%%%%%%%%%%%%%%%%%%%%%%%%%%%%%%%%%%%%%%%%%%%%%%%%%%%%%%%%%%%%%%%

\begin{titlepage}
	\centering
    \vspace*{0.5 cm}
   % \includegraphics[scale = 0.075]{bsulogo.png}\\[1.0 cm]	% University Logo
\begin{center}    \textsc{\Large   ECE 351 - Section \#52 }\\[2.0 cm]	\end{center}% University Name
	\textsc{\Large Block Diagrams and System Stability  }\\[0.5 cm]				% Course Code
	\rule{\linewidth}{0.2 mm} \\[0.4 cm]
	{ \huge \bfseries \ Lab 7}\\
	\rule{\linewidth}{0.2 mm} \\[1.5 cm]
	
	\begin{minipage}{0.4\textwidth}
		\begin{flushleft} \large
		%	\emph{Submitted To:}\\
		%	Name\\
          % Affiliation\\
           %contact info\\
			\end{flushleft}
			\end{minipage}~
			\begin{minipage}{0.4\textwidth}
            
			\begin{flushright} \large
			\emph{Submitted By :} \\
			Treshur Allen  
		\end{flushright}
           
	\end{minipage}\\[2 cm]
	
	%\includegraphics[scale = 0.5]{PICMathLogo.png}
    
    
    
    
	
\end{titlepage}

%%%%%%%%%%%%%%%%%%%%%%%%%%%%%%%%%%%%%%%%%%%%%%%%%%%%%%%%%%%%%%%%%%%%%%%%%%%%%%%%%%%%%%%%%

\tableofcontents
\pagebreak

%%%%%%%%%%%%%%%%%%%%%%%%%%%%%%%%%%%%%%%%%%%%%%%%%%%%%%%%%%%%%%%%%%%%%%%%%%%%%%%%%%%%%%%%%
\renewcommand{\thesection}{\arabic{section}}
\section{Introduction}
 
The purpose of this lab was to become familiar with the Laplace-domain by using the factored form of the found transfer function to determine the systems stability. We were able to do this by using the scipy.signal library functions. These allowed us to perform the necessary math such as finding the zero, pull and gain in the functions. We performed these equations twice, once on the first block diagrams open loop function, then again on the closed loop.

\section{Equations}

Here are the equations used in the designated sections. along with their definitions: 

\subsection{Part 1 Task 1}

\begin{equation*}
G(s) = \frac{s + 9}{(s - 8)(s + 2)(s + 4)}
\end{equation*}
\begin{equation*}
A(s) = \frac {s + 4}{(s + 3)(s + 1)} 
\end{equation*}

\begin{equation*}
B(s) = (s + 14)(s + 12)
\end{equation*}

\subsection{Part 1 Task 3}

\begin{equation*}
\frac{y(t)}{x(t)} = A(s) \cdot G(s)
\end{equation*}

\begin{equation*}
=> \frac{(s+4)(s+9)}{(s+3)(s+1)(s-8)(s+2)(s+4)}
\end{equation*}

\subsection{Part 2 Task 1 & 2}

\begin{equation*}
\frac{A \cdot G}{1 + B \cdot G}
\end{equation*}

\begin{equation*}\large
\frac{\frac{numA \cdot numG}{denA \cdot denG}}{1 + \frac{numB \cdot numG}{denB \cdot denG}}
\end{equation*}

\section{Methodology}

The way I went about solving this lab was after hand calculating the transfer functions I was able to put them in factored form to check for gain, zeros, and polls. To calculate these factors along with the step response to the function I was also able to use the given commands from the lab handout like: sig.tf2zpk(), sig.convolve() and sig.step(). These were used with arrays representing the coefficients of the polynomials in the numerator and denominator. Here are some examples of how I used those functions in my code.
\begin{lstlisting}[language=Python]
def G():
    num = [0, 0, 1, 9]
    den = [1, -2, -40, -64]
    print('\nG(s) zero, poll, gain: ', sig.tf2zpk(num, den))
    return 

numer = sig.convolve([1, 4], [1, 9])
denom = sig.convolve([1, 4, 3], [1, -2, -40, -64])

resp = [numer, denom]

t,h = sig.step(resp)

\end{lstlisting}

\section{Results}

For the results I acquired in this lab, it was mainly the output on the console. This was the answers to the poll, zeros, and gain along with the final convolutions to get the transfer functions. 

\section{Error Analysis}

When working through this lab the only difficulties I think I had was making sure my equations were correct to insure that the graphs would turn out the way they were supposed to.

\section{Questions}

\subsection{Part 1 Task 4} 
The equation I came up with is unstable because the zeros of the system are not relative to one constant, they are sporadic therefore making the system unstable 

\subsection{Part 1 Task 6} 

As time goes on the systems graph takes off like an exponential therefore the answer I concluded in task 4 does match the graph produced in task 5

\subsection{Part 2 Task 3} 
The equation although is different from the open loop, it is still unstable for the same reasons. 

\subsection{Part 2 Task 5} 

I would say that after looking at the graph, it is similar to the one from part 1 which also had an unstable function therefore the conclusion from task 4 does align with that of task 3.

\section{Conclusion}

I learned how to find the polls, zeros and gain of functions and better understood how the sig.step function is used for plotting. But when trying to print the results of the step response you receive a very large array of elements. 

\newpage


\begin{thebibliography}{111}

  \bibitem{ACMT}
https://docs.scipy.org/doc/scipy-0.14.0/reference/generated/scipy.signal.tf2zpk.html



\end{thebibliography}
\end{document}

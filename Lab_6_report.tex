% % % % % % % % % % % % % % % % % % % % % % % % % % % % % % % %
%                                                             %
% Treshur Allen                                               %
% ECE 351 - 52                                                %
% Lab 6                                                       %
% October 14, 2021                                            %
% This file contains the tex file for lab 6 with report and   %
% questions.                                                  %
% % % % % % % % % % % % % % % % % % % % % % % % % % % % % % % %

\documentclass{article}
\usepackage[utf8]{inputenc}


\title{ECE351 Lab 6 Report}
\author{Treshur Allen }
\date{October 14, 2021}

\begin{document}

\pagebreak

\maketitle

\section{Part 1:}

\subsection{Description:}

\begin{verbatim}
The first part of this lab was to incorporate what we had calculated in the prelab
we were to type what equation we had come up with from the prelab.
For the second task we were to use the .step() function from the signals 
lib. after completing that task we were to plot and compare the equations. 

\end{verbatim}

\subsection{Task Codes:}

\begin{verbatim}
steps = 1e-2 #step size
t = np.arange(0,2 +steps, steps)
#---------------------Part 1 task 1 FUNCTIONS --------------------------
def stepFunc(t):
   
    u = np.zeros(t.shape)
    
    for i in range(len(t)):
        if t[i] <= 0:  
            u[i] = 0
        else:
            u[i] = 1
    return u

def prelab_y(t):
    y = 6.93*np.sin(t + 89.8) * stepFunc(t) 
    return y

#---------------------Part 1 task 2 FUNCTIONS --------------------------
num = [1, 10, 24]
den = [1, 6, 12]
#H = sig.TransferFunction(num,den)
#h = sig.step(num,den)

w,h = sig.freqs(num, den)

#---------------------PLOTS-------------------------

plt.figure(figsize = (10, 7))
plt.subplot(2, 1, 1)
plt.plot(t, prelab_y(t))
plt.grid()
plt.ylabel('y(t)')
plt.title('Part 1 Task 1')


plt.subplot(2 , 1, 2)
plt.plot(w, h)
plt.grid()
plt.ylabel('H(s)')
plt.title('Part 1 Task 2')
plt.show()

#------------------------Part 1 Task 3--------------------
print('\nThe coefficients from partial fraction Part 1:\n', sig.residue(num,den))

\end{verbatim}

\subsection{Explanation:}

\begin{verbatim}
Within my code I found that you can plot the transfer function by using the freqs() function 
from the signals lib. 
\end{verbatim}

\maketitle

\section{Part 2:}

\subsection{Description:}

\begin{verbatim}
The purpose of this part of the lab was to test what we had just learned in the 
previos part by repeating the same process. 
\end{verbatim}

\subsection{Task Codes:}

\begin{verbatim}
#---------------------Part 2 Task 1---------------------------

denom = [0, 0, 0, 0, 0, 25250]
numer = [1, 18, 218, 2036, 9085, 25250]

print('\nThe coefficients from partial fraction Part 2:\n', sig.residue(numer,denom))

#---------------------part 2 task 2------------------------

d,g = sig.freqs(numer, denom)

plt.figure(figsize = (10, 7))
plt.subplot(2, 1, 1)
plt.plot(d,g)
plt.grid()
plt.ylabel('H(s)')
plt.title('Part 2 Task 2')

#--------------------------------Part 2 task 3-------------------------

h = sig.step(numer,denom)


plt.subplot(2, 1, 2)
plt.plot(t,h)
plt.grid()
plt.ylabel('H(s)')
plt.title('Part 2 Task 3')

\end{verbatim}

\subsection{Explanation:}

\begin{verbatim}
Within my code for this part it was vary similar to the first part 
in the sense that I was able to plot the transfer function by using 
the freqs() function from the lib. we also had to use the residue()
function that is stated in the lab manual this was used pretty similarly 
to how I used the .step() function but needed a different set of denominator
values. 
\end{verbatim}

\maketitle

\section{Questions:}

\subsection{For a non-complex pole-residue term, you can still use the cosine method, explain why this
works.}

\begin{verbatim}
You can use the cosine method on these non-complex terms because when using cosine rather
than sine, you are dealing with non-imaginary terms which makes finding this process easier. 
\end{verbatim}

\subsection{Leave any feedback on the clarity of the expectations, instructions, and deliverable's.:}

\begin{verbatim}
Nothing to report 
\end{verbatim}

\end{document}
% % % % % % % % % % % % % % % % % % % % % % % % % % % % % % % %
%                                                             %
% Treshur Allen                                               %
% ECE 351 - 52                                                %
% Lab 8                                                       %
% October 28, 2021                                            %
% This file contains the tex file for lab 8 with report and   %
% questions.                                                  %
% % % % % % % % % % % % % % % % % % % % % % % % % % % % % % % %

%%%%%%%%%%%%%%%%%%%%%%%%%%%%%%%%%%%%%%%%%%%
%%% DOCUMENT PREAMBLE %%%
\documentclass[12pt]{report}
\usepackage[english]{babel}
%\usepackage{natbib}
\usepackage{url}
\usepackage[utf8x]{inputenc}
\usepackage{amsmath}
\usepackage{graphicx}
\graphicspath{{images/}}
\usepackage{parskip}
\usepackage{fancyhdr}
\usepackage{vmargin}
\usepackage{listings}
\usepackage{hyperref}
\usepackage{xcolor}

\definecolor{codegreen}{rgb}{0,0.6,0}
\definecolor{codegray}{rgb}{0.5,0.5,0.5}
\definecolor{codeblue}{rgb}{0,0,0.95}
\definecolor{backcolour}{rgb}{0.95,0.95,0.92}

\lstdefinestyle{mystyle}{
    backgroundcolor=\color{backcolour},   
    commentstyle=\color{codegreen},
    keywordstyle=\color{codeblue},
    numberstyle=\tiny\color{codegray},
    stringstyle=\color{codegreen},
    basicstyle=\ttfamily\footnotesize,
    breakatwhitespace=false,         
    breaklines=true,                 
    captionpos=b,                    
    keepspaces=true,                 
    numbers=left,                    
    numbersep=5pt,                  
    showspaces=false,                
    showstringspaces=false,
    showtabs=false,                  
    tabsize=2
}
 
\lstset{style=mystyle}

\setmarginsrb{3 cm}{2.5 cm}{3 cm}{2.5 cm}{1 cm}{1.5 cm}{1 cm}{1.5 cm}

\title{Lab 8}								
% Title
\author{ Treshur Allen}						
% Author
\date{October 28, 2021}
% Date

\makeatletter
\let\thetitle\@title
\let\theauthor\@author
\let\thedate\@date
\makeatother

\pagestyle{fancy}
\fancyhf{}
\rhead{\theauthor}
\lhead{\thetitle}
\cfoot{\thepage}
%%%%%%%%%%%%%%%%%%%%%%%%%%%%%%%%%%%%%%%%%%%%
\begin{document}

%%%%%%%%%%%%%%%%%%%%%%%%%%%%%%%%%%%%%%%%%%%%%%%%%%%%%%%%%%%%%%%%%%%%%%%%%%%%%%%%%%%%%%%%%

\begin{titlepage}
	\centering
    \vspace*{0.5 cm}
   % \includegraphics[scale = 0.075]{bsulogo.png}\\[1.0 cm]	% University Logo
\begin{center}    \textsc{\Large   ECE 351 - Section \#52 }\\[2.0 cm]	\end{center}% University Name
	\textsc{\Large Fourier Series Approximation of a Square Wave  }\\[0.5 cm]				% Course Code
	\rule{\linewidth}{0.2 mm} \\[0.4 cm]
	{ \huge \bfseries \ Lab 8}\\
	\rule{\linewidth}{0.2 mm} \\[1.5 cm]
	
	\begin{minipage}{0.4\textwidth}
		\begin{flushleft} \large
		%	\emph{Submitted To:}\\
		%	Name\\
          % Affiliation\\
           %contact info\\
			\end{flushleft}
			\end{minipage}~
			\begin{minipage}{0.4\textwidth}
            
			\begin{flushright} \large
			\emph{Submitted By :} \\
			Treshur Allen  
		\end{flushright}
           
	\end{minipage}\\[2 cm]
	
	%\includegraphics[scale = 0.5]{PICMathLogo.png}
    
    
    
    
	
\end{titlepage}

%%%%%%%%%%%%%%%%%%%%%%%%%%%%%%%%%%%%%%%%%%%%%%%%%%%%%%%%%%%%%%%%%%%%%%%%%%%%%%%%%%%%%%%%%

\tableofcontents
\pagebreak

%%%%%%%%%%%%%%%%%%%%%%%%%%%%%%%%%%%%%%%%%%%%%%%%%%%%%%%%%%%%%%%%%%%%%%%%%%%%%%%%%%%%%%%%%
\renewcommand{\thesection}{\arabic{section}}
\section{Introduction}
 
The purpose of this lab was to learn how we can use Fourier series to approximate a periodic wave in the time domain. In order for us to do this successfully we Incorporated our prelab which was to find the general expression for the coefficients of the given Fourier series. 

\section{Equations}

Here are the equations used in the designated sections. along with their definitions: 

\subsection{Task 1}

\begin{equation*}
A(k) = 0
\end{equation*}
\begin{equation*}
B(k) = \frac{2}{n \cdot \pi}  \cdot [1 - cos(k \cdot \pi)]
\end{equation*}

\section{Methodology}
\subsection{Task 1}
The way I went about solving this lab was typing the answers from the prelab as functions that could later be used to calculate their values at the requested values of a0, a1, b1, b2, and b3.
\begin{lstlisting}[language=Python]
def b(k):
    b = (2 / (k * 3.14159)) * (1 - np.cos(k * 3.14159))
    return b

\end{lstlisting}

\subsection{Task 2}
When it came to plotting the summations, I found it easiest to use numpys linespace to get the correct scope of 0s < t < 20s with a number of 1000 steps taken for better resolution. Then I created an array initialized to zeros to hold all the summation points. lastly I incorporated a for loop to step through all the summation points and put them in the array. I repeated this process for each specified value of N.

\begin{lstlisting}[language=Python]
x = np.linspace(0, 20, 1000) 
Psi1 = 0*x # now Psi is an array of zeros

N = 1
# second input of range is N+1 since our index n satisfies 1 <= n < N+1
# third input makes n increment by 1 each loop (the default)
for n in range(1, N+1, 1):
    Psi1 += b(n)

plt.figure(figsize = (10, 10))

plt.subplot(3, 1, 1)
plt.plot(x, Psi1 )
plt.grid()
plt.ylabel('N = 1')
plt.title('Sumation plots 1-3')

\end{lstlisting}

\section{Results}

For the results I acquired in this lab consisted of the outputs for the a(k) and b(k) functions along with the six plots for the summation N = {1, 3, 15, 50, 150, 1500}. These will be included in the appendix file. 

\section{Error Analysis}

When working through this lab I originally had an issue getting my graphs to plot the array correctly. It was giving me a constant flat line of what the actual sum was for the N value rather than plotting the approximation. I fixed this by taking the x axis time values into account so using the array that I defined that held the values 0s < t < 20s. 

\section{Questions}

1. Is x(t) an even or an odd function? Explain why.
this is an odd function, we know this by looking at the symmetry for the period. if it is symmetrical or a cosine type wave, we would have an even function. But what makes our function unique is that we are using a cosine function but without the 50 percent duty cycle and our translations it turns our normally even cosine function into an odd function.

2. Based on your results from Task 1, what do you expect the values of a2, a3, . . . , an to be? Why?
    For all A(k) values, it is expected to be zero.

3. How does the approximation of the square wave change as the value of N increases? In what way does the Fourier series struggle to approximate the square wave?
    As the value of N increases the trough of the graphs increase and the crests have more fluctuation or waves, but look more like a smooth curve because there are so many of these squiggles. 

4. What is occurring mathematically in the Fourier series summation as the value of N increases?
    Mathematically as N increases, our approximation should become more accurate because you are calculating more points giving you more precision. 

5. Leave any feedback on the clarity of lab tasks, expectations, and deliverables.
    I have nothing to add, I enjoyed this lab. :)

\section{Conclusion}

Now that I have a better understanding of how to plot functions, the main skill I learned from this lab was how to plot summations, along with becoming more comfortable with for loops. 

\newpage


\begin{thebibliography}{111}

  \bibitem{ACMT}
https://docs.scipy.org/doc/scipy-0.14.0/reference/index.html




\end{thebibliography}
\end{document}

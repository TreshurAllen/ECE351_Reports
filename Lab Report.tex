% % % % % % % % % % % % % % % % % % % % % % % % % % % % % % % %
%                                                             %
% Treshur Allen                                               %
% ECE 351 - 52                                                %
% Lab 11                                                      %
% December 9, 2021                                            %
% This file contains the tex file for lab 12 with report and  %
% questions.                                                  %
% % % % % % % % % % % % % % % % % % % % % % % % % % % % % % % %

%%%%%%%%%%%%%%%%%%%%%%%%%%%%%%%%%%%%%%%%%%%
%%% DOCUMENT PREAMBLE %%%
\documentclass[12pt]{report}
\usepackage[english]{babel}
%\usepackage{natbib}
\usepackage{url}
\usepackage[utf8x]{inputenc}
\usepackage{amsmath}
\usepackage{graphicx}
\graphicspath{{images/}}
\usepackage{parskip}
\usepackage{fancyhdr}
\usepackage{vmargin}
\usepackage{listings}
\usepackage{hyperref}
\usepackage{xcolor}
\usepackage{graphicx}
\graphicspath{ {./images/} }

\definecolor{codegreen}{rgb}{0,0.6,0}
\definecolor{codegray}{rgb}{0.5,0.5,0.5}
\definecolor{codeblue}{rgb}{0,0,0.95}
\definecolor{backcolour}{rgb}{0.95,0.95,0.92}

\lstdefinestyle{mystyle}{
    backgroundcolor=\color{backcolour},   
    commentstyle=\color{codegreen},
    keywordstyle=\color{codeblue},
    numberstyle=\tiny\color{codegray},
    stringstyle=\color{codegreen},
    basicstyle=\ttfamily\footnotesize,
    breakatwhitespace=false,         
    breaklines=true,                 
    captionpos=b,                    
    keepspaces=true,                 
    numbers=left,                    
    numbersep=5pt,                  
    showspaces=false,                
    showstringspaces=false,
    showtabs=false,                  
    tabsize=2
}
 
\lstset{style=mystyle}

\setmarginsrb{3 cm}{2.5 cm}{3 cm}{2.5 cm}{1 cm}{1.5 cm}{1 cm}{1.5 cm}

\title{Lab 11}								
% Title
\author{ Treshur Allen}						
% Author
\date{December 9, 2021}
% Date

\makeatletter
\let\thetitle\@title
\let\theauthor\@author
\let\thedate\@date
\makeatother

\pagestyle{fancy}
\fancyhf{}
\rhead{\theauthor}
\lhead{\thetitle}
\cfoot{\thepage}
%%%%%%%%%%%%%%%%%%%%%%%%%%%%%%%%%%%%%%%%%%%%
\begin{document}

%%%%%%%%%%%%%%%%%%%%%%%%%%%%%%%%%%%%%%%%%%%%%%%%%%%%%%%%%%%%%%%%%%%%%%%%%%%%%%%%%%%%%%%%%

\begin{titlepage}
	\centering
    \vspace*{0.5 cm}
   % \includegraphics[scale = 0.075]{bsulogo.png}\\[1.0 cm]	% University Logo
\begin{center}    \textsc{\Large   ECE 351 - Section \#52 }\\[2.0 cm]	\end{center}% University Name
	\textsc{\Large Final Lab }\\[0.5 cm]				% Course Code
	\rule{\linewidth}{0.2 mm} \\[0.4 cm]
	{ \huge \bfseries \ Lab 12}\\
	\rule{\linewidth}{0.2 mm} \\[1.5 cm]
	
	\begin{minipage}{0.4\textwidth}
		\begin{flushleft} \large
		%	\emph{Submitted To:}\\
		%	Name\\
          % Affiliation\\
           %contact info\\
			\end{flushleft}
			\end{minipage}~
			\begin{minipage}{0.4\textwidth}
            
			\begin{flushright} \large
			\emph{Submitted By :} \\
			Treshur Allen  
		\end{flushright}
           
	\end{minipage}\\[2 cm]
	
	%\includegraphics[scale = 0.5]{PICMathLogo.png}
    
    
    
    
	
\end{titlepage}

%%%%%%%%%%%%%%%%%%%%%%%%%%%%%%%%%%%%%%%%%%%%%%%%%%%%%%%%%%%%%%%%%%%%%%%%%%%%%%%%%%%%%%%%%

\tableofcontents
\pagebreak

%%%%%%%%%%%%%%%%%%%%%%%%%%%%%%%%%%%%%%%%%%%%%%%%%%%%%%%%%%%%%%%%%%%%%%%%%%%%%%%%%%%%%%%%%
\renewcommand{\thesection}{\arabic{section}}
\section{Introduction}
 
This being the final lab, its main purpose was to test all the skills we were to acquire throughout the 
course. The first thing that was different and stood out was that the tasks and prompt were more open ended,
we were being tested on how we could answer these problems with certain parameters with the knowledge we gained. 

\section{Equations}

When it comes to the equations used in this lab the main ones I used that I came up with was the 
equation format for a band pass filter, how to calculate magnitude and phase. 

\subsection{Band Pass}
\begin{equation*}
\frac{1/RC\cdot S}{S^2 + 1/RC\cdot S + 1/LC }
\end{equation*}

\subsection{Magnitude}
\begin{equation*}
20*log((w / (R*C)) / sqrt( w^4 + ((1/R*C)^2 - (2/C*C))*w^2 + (1/L*C)^2))
\end{equation*}

\subsection{Phase}
\begin{equation*}
(pi / 2) - arctan( (w /R*C)/(-w^2 + 1/L*C))
\end{equation*}

\section{Methodology}
A lot of my methodology is outlined within the lab itself, but in summary the first thing I did was add the 
code that was provided in the appendix, with a little modification I got the correct graph for the sample frequency and 
next created my RLC circuit for the filter. I modeled this after a previous lab where we used a band pass filter to get our 
set range which was a combination of what could be considered in between high and low frequencies. My next plan of action was 
to start working on the other task requirements down the line working task by task. Again most of the code I used was based on 
what I had accomplished in previous labs, like the FFT and Bode plot functions. Below is a photo of the hand written calculations I 
performed to get my values for the filter. 

\subsection{Band Pass Calculations:}
\includegraphics{hand_calc_pt1.JPG}
\includegraphics{hand_calc_pt2.JPG}

\section{Results}
The results for this lab was the graphs that were produced by each of the tasks which are provided below:

\subsection{Input Signal:}
\includegraphics{input.png}

\subsection{Magnitude and Phase:}
\includegraphics{mag_phi.JPG}

\subsection{Bode plot:}
\includegraphics{bode.png}

\subsection{Final Filter:}
\includegraphics{final.png}

\section{Error Analysis}

When I first started working on this lab I had the most trouble on understanding the first task which after looking 
through the other tasks I realized I was over thinking what was being asked. After working through the files that were provided outside the lab manual it helped provide some clarity on where to start and what the purpose / end result of the lab should be. 

\section{Questions}

1. Earlier this semester, you were asked what you personally wanted to get out of taking this
course. Do you feel like that personal goal was met? Why or why not?
I feel that I have learned great ways to post the work that I've done so my future employers can see, 
I also feel that my python coding skills have been greatly enhanced. 

2. Please fill out the course feedback survey, I will read every word and very much appreciate
the feedback.
**thumbs up**

3. Good luck in the rest of your education and career!
Thank you Kate!

\section{Conclusion}
I can conclude that this lab was by far more difficult than any of the other we had done thus far, but at the same time I feel that it was a challenging final test for us to really see if we can incorporate what we had learned. 


\newpage


\begin{thebibliography}{111}

  \bibitem{ACMT}
https://docs.scipy.org/doc/scipy-0.14.0/reference/index.html




\end{thebibliography}
\end{document}
